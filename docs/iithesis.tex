% Opcje klasy 'iithesis' opisane sa w komentarzach w pliku klasy. Za ich pomoca
% ustawia sie przede wszystkim jezyk i rodzaj (lic/inz/mgr) pracy, oraz czy na
% drugiej stronie pracy ma byc skladany wzor oswiadczenia o autorskim wykonaniu.
\documentclass[declaration,shortabstract,polish,inz]{iithesis}
\usepackage[utf8]{inputenc}
\usepackage{fancyhdr}
\usepackage[toc]{appendix}
\usepackage{bookmark}
\usepackage{enumerate}
\usepackage{csvsimple}

\usepackage{color}
\definecolor{bluekeywords}{rgb}{0.13,0.13,1}
\definecolor{greencomments}{rgb}{0,0.5,0}
\definecolor{redstrings}{rgb}{0.9,0,0}

\definecolor{lightgray}{rgb}{.9,.9,.9}
\definecolor{darkgray}{rgb}{.4,.4,.4}
\definecolor{purple}{rgb}{0.65, 0.12, 0.82}

\usepackage{listings}

\lstdefinelanguage{JavaScript}{
  keywords={typeof, new, true, false, catch, function, return, null, catch, switch, var, if, in, while, do, else, case, break},
  keywordstyle=\color{blue}\bfseries,
  ndkeywords={class, export, boolean, throw, implements, import, this},
  ndkeywordstyle=\color{darkgray}\bfseries,
  identifierstyle=\color{black},
  sensitive=false,
  comment=[l]{//},
  morecomment=[s]{/*}{*/},
  commentstyle=\color{purple}\ttfamily,
  stringstyle=\color{red}\ttfamily,
  morestring=[b]',
  morestring=[b]"
}

\lstset{language=JavaScript,
  showspaces=false,
  showtabs=false,
  breaklines=true,
  showstringspaces=false,
  breakatwhitespace=true,
  escapeinside={(*@}{@*)},
  commentstyle=\color{greencomments},
  keywordstyle=\color{bluekeywords},
  stringstyle=\color{redstrings},
  basicstyle=\ttfamily
}

\pagestyle{fancy}
\fancyhf{}
\fancyfoot[CO,CE]{\thepage}
\fancyhead[RE]{\leftmark}
\fancyhead[LO]{\rightmark}
\renewcommand{\chaptermark}[1]{\markboth{Wiktor Adamski, Instytut Informatyki UWr}{}}
\renewcommand{\sectionmark}[1]{\markright{Implementacja protokołu LSP dla wybranego środowiska zintegrowanego}}

\polishtitle{Implementacja protokołu LSP\fmlinebreak dla wybranego środowiska zintegrowanego}
\englishtitle{Implementation of a LSP protocol\fmlinebreak for chosen IDE}

\author{Wiktor Adamski}
\advisor{dr Wiktor Zychla}
\date{30 stycznia 2018 r.} % Data zlozenia pracy
\transcriptnum{272220} % Numer indeksu
\advisorgen{dr Wiktora Zychli} % Nazwisko promotora w dopelniaczu

\polishabstract{}

\englishabstract{}

\begin{document}

\chapter{Preliminaria}
Aby zrozumieć jak działa dostarczony serwer, należy wpierw zrozumieć architekturę rozszerzenia w systemie edytora Visual Studio Code, a także na czym polega omawiany protokół. W tym rozdziale poruszone zostaną:

\begin{itemize}
    \item Struktura wtyczki rozszerzającej działanie edytora.
    \item Opis protokołu LSP.
    \item Visual Studio a Visual Studio Code.
    \item Parser Luaparse.
\end{itemize}

\section{Struktura rozszerzenia Visual Studio Code}
Sercem każdego rozszerzenia jest plik \texttt{package.json}, który przechowuje informacje na temat autora pakietu, warunki jego uruchomienia, a także wszystkie jego zależności:

\begin{lstlisting}
    {
        "name": "lua-lang",
        "description": "Lua language support",
        "author": "Wiktor Adamski",
        "license": "MIT",
        "version": "0.0.1",
        "engines": {
            "vscode": "^1.16.0"
        },
        "categories": [
            "Languages"
        ],
        "activationEvents": [
            "onLanguage:lua"
        ],
        "main": "./out/src/extension",
        "contributes": {
            "languages": [
                {
                    "id": "lua",
                    "aliases": [
                        "Lua",
                        "lua"
                    ],
                    "extensions": [
                        ".lua",
                        ".p8",
                        ".rockspec"
                    ],
                    "configuration": "./language-configuration.json"
                }
            ],
        },
        "dependencies": {
            "vscode": "^1.1.5",
            "vscode-languageclient": "^3.4.2"
        }
    }
\end{lstlisting}

O ile serwer LSP może być napisany w dowolnym języku, część bezpośrednio łącząca się z edytorem (aktualna wtyczka) musi być w języku JavaScript dla środowiska uruchomieniowego node.js. 

Duża część kodu który jest wspólny dla wszystkich rozszerzeń jest możliwa do automatycznego stworzenia przez generator kodu Yeoman. Proste polecenie \texttt{yo code} przeprowadzi nas przez kreator wtyczek i utworzy dodatkowe pliki konfiguracyjne, które pozwolą korzystać z edytora VS Code jako środowiska developerskiego.

\section{Protokół Language Server Protocol}
Protokół LSP \cite{docs} jest specjalizacją protokołu JSON-RPC, który przesyła dane między stronami komunikacji za pomocą obiektów JSON. Klient (edytor kodu) wysyła zapytania do serwera (program wspomagający) odpowiadające różnym akcjom podejmowanym przez programistę, np. zapytanie się o miejsce deklaracji danej zmiennej. Pierwszą wiadomością w trakcie połączenia jest wymiana możliwości zarówno klienta (np. czy edytor wspiera przemianowanie zmiennej), jak i serwera (np. wskazanie definicji danego symbolu lub automatyczne uzupełnianie pisanego tekstu). Twórcy protokołu udostępnili bibliotekę bibliotekę korzystanie z niego w języku TypeScript. Poniżej przedstawiam przykładowe zapytanie klienta:

\begin{lstlisting}
    {
        "jsonrpc": "2.0",
        "id": 1,
        "method": "textDocument/didOpen",
        "params": {
            ...
        }
    }
\end{lstlisting}

\section{Visual Studio (Code)}
Wiele osób nie rozróżnia od siebie dwóch produktów Microsoftu. Visual Studio to zintegrowane środowisko programistyczne, nastawione głównie na pisanie programów w języku C\#. Visual Studio Code jest natomiast otwartoźródłowym edytorem kodu opartym na silniku renderującym Electron od firmy Github, co sprawia, że jest on dosyć podobny do edytora Atom (również pod względem metodologii wtyczek). 

\section{Biblioteka Luaparse i drzewa rozbioru}
Aby dostarczać jakiekolwiek sensowne informacje na temat kodu, potrzebne jest jego sparsowanie. Zajmuje się tym biblioteka Luaparse \cite{luaparse}, która produkuje abstrakcyjne drzewa rozbioru programów napisanych w języku Lua. Drzewa reprezentowane za pomocą obiektów JavaScript są inspirowane na specyfikacji Mozilla Parser API. Przykładowo wyrażenie:

\begin{lstlisting}[language={[5.3]Lua}]
foo = "bar"
\end{lstlisting}
zostanie przełożone na drzewo:

\begin{lstlisting}
    {
        "type":"Chunk",
        "body":[{
            "type":"AssignmentStatement",
            "variables":[{
                "type":"Identifier",
                "name":"foo",
                "loc":{
                    "start":{ "line":1, "column":0 },
                    "end":{ "line":1, "column":3 }
                }
            }],
            "init":[{
                "type":"StringLiteral",
                "value":"bar",
                "raw":"\"bar\"",
                "loc":{
                    "start":{ "line":1, "column":6 },
                    "end":{ "line":1, "column":11 }
                }
            }],
            "loc":{
                "start":{ "line":1, "column":0 },
                "end":{ "line":1, "column":11 }
            }
        }],
        "loc":{
            "start":{ "line":1, "column":0 },
            "end":{ "line":1, "column":11 }
        },
        "comments":[]
    }
\end{lstlisting}


\begin{thebibliography}{1}
\bibitem{docs} Microsoft \href{https://microsoft.github.io/language-server-protocol/specification#initialize}{Language Server Protocol documentation}, 2018.
\bibitem{luaparse} Oskar Schöldström \href{https://oxyc.github.io/luaparse/}{Luaparse}, 2013 - 2017
\bibitem{lsp_sample} Microsoft \href{https://code.visualstudio.com/docs/extensions/example-language-server}{Creating Language Servers for Visual Studio Code}, 2018.
\bibitem{lua_lib} Roberto Ierusalimschy, Luiz Henrique de Figueiredo, Waldemar Celes \href{https://www.lua.org/manual/5.3/manual.html}{Lua 5.3 Reference Manual}, 2015 - 2017

\end{thebibliography}

\renewcommand\appendixtocname{Dodatki}
\bookmarksetupnext{level=part}
\begin{appendices}
\addtocontents{toc}{\protect\setcounter{tocdepth}{1}}
\makeatletter
\addtocontents{toc}{
  \begingroup
  \let\protect\l@chapter\protect\l@section
  \let\protect\l@section\protect\l@subsection
}
\makeatother
  \chapter{Instrukcja uruchomienia rozszerzenia}
Tu będzie jakaś wystrzałowa instrukcja uruchomienia, że aż wszystkim gacie pospadają z wrażenia
\addtocontents{toc}{\endgroup}
\end{appendices}

\end{document}