% Opcje klasy 'iithesis' opisane sa w komentarzach w pliku klasy. Za ich pomoca
% ustawia sie przede wszystkim jezyk i rodzaj (lic/inz/mgr) pracy, oraz czy na
% drugiej stronie pracy ma byc skladany wzor oswiadczenia o autorskim wykonaniu.
\documentclass[declaration,shortabstract,polish,inz]{iithesis}
\usepackage[utf8]{inputenc}
\usepackage{fancyhdr}
\usepackage[toc]{appendix}
\usepackage{bookmark}
\usepackage{enumerate}
\usepackage{csvsimple}

\usepackage{color}
\definecolor{bluekeywords}{rgb}{0.13,0.13,1}
\definecolor{greencomments}{rgb}{0,0.5,0}
\definecolor{redstrings}{rgb}{0.9,0,0}

\definecolor{lightgray}{rgb}{.9,.9,.9}
\definecolor{darkgray}{rgb}{.4,.4,.4}
\definecolor{purple}{rgb}{0.65, 0.12, 0.82}

\usepackage{listings}

\lstdefinelanguage{JavaScript}{
  keywords={typeof, new, true, false, catch, function, return, null, catch, switch, var, if, in, while, do, else, case, break},
  keywordstyle=\color{blue}\bfseries,
  ndkeywords={class, export, boolean, throw, implements, import, this},
  ndkeywordstyle=\color{darkgray}\bfseries,
  identifierstyle=\color{black},
  sensitive=false,
  comment=[l]{//},
  morecomment=[s]{/*}{*/},
  commentstyle=\color{purple}\ttfamily,
  stringstyle=\color{red}\ttfamily,
  morestring=[b]',
  morestring=[b]"
}

\lstset{language=JavaScript,
  showspaces=false,
  showtabs=false,
  breaklines=true,
  showstringspaces=false,
  breakatwhitespace=true,
  escapeinside={(*@}{@*)},
  commentstyle=\color{greencomments},
  keywordstyle=\color{bluekeywords},
  stringstyle=\color{redstrings},
  basicstyle=\ttfamily
}

\pagestyle{fancy}
\fancyhf{}
\fancyfoot[CO,CE]{\thepage}
\fancyhead[RE]{\leftmark}
\fancyhead[LO]{\rightmark}
\renewcommand{\chaptermark}[1]{\markboth{Wiktor Adamski, Instytut Informatyki UWr}{}}
\renewcommand{\sectionmark}[1]{\markright{Implementacja protokołu LSP dla wybranego środowiska zintegrowanego}}

\polishtitle{Implementacja protokołu LSP\fmlinebreak dla wybranego środowiska zintegrowanego}
\englishtitle{Implementation of a LSP protocol\fmlinebreak for chosen IDE}

\author{Wiktor Adamski}
\advisor{dr Wiktor Zychla}
\date{30 stycznia 2018 r.} % Data zlozenia pracy
\transcriptnum{272220} % Numer indeksu
\advisorgen{dr Wiktora Zychli} % Nazwisko promotora w dopelniaczu

\polishabstract{Visual Studio Code to na pierwszy rzut oka prosty edytor kodu, jednakże system rozszerzeń pozwala rozbudować go do pełnoprawnego środowiska programistycznego. Ponieważ istnieje wiele edytorów, a każdy z nich posiadał swój interfejs programistyczny, powstał protokół Language Server Protocol (LSP), który umożliwia napisanie logiki wspomagającej pisanie programu raz i użycie jej w wielu edytorach. Tematem tej pracy jest implementacja rozszerzenia do edytora kodu wykorzystującego ww. protokół.}

\englishabstract{Visual Studio Code at first seems like a simple code editor, though its extension system allows to expand it into full-featured IDE. As there are many editors, and each of them has its own application interface, Language Server Protocol (LSP) was created, to allow writing helper logic once and using it in many editors. The topic of this thesis is the implementation of code editor extension which uses said protocol.}

\begin{document}

\chapter{Preliminaria}
Aby zrozumieć jak działa dostarczony serwer, należy wpierw zrozumieć architekturę rozszerzenia w systemie edytora Visual Studio Code, a także na czym polega omawiany protokół. W tym rozdziale poruszone zostaną:

\begin{itemize}
    \item Struktura wtyczki rozszerzającej działanie edytora.
    \item Opis protokołu LSP.
\end{itemize}

\section{Struktura rozszerzenia Visual Studio Code}
Sercem każdego rozszerzenia jest plik \texttt{package.json}, który przechowuje informacje na temat autora pakietu, warunki jego uruchomienia, a także wszystkie jego zależności. O ile sam serwer może być napisany w dowolnym języku, część bezpośrednio łącząca się z edytorem musi być napisana w języku JavaScript dla środowiska uruchomieniowego node.js. 

Duża część kodu który jest wspólny dla wszystkich rozszerzeń jest możliwa do automatycznego stworzenia przez generator kodu Yeoman. Proste polecenie \texttt{yo code} przeprowadzi nas przez kreator wtyczek i utworzy dodatkowe pliki konfiguracyjne, które pozwolą korzystać z edytora VS Code jako środowiska developerskiego.

\section{Protokół Language Server Protocol}
Protokół LSP jest specjalizacją protokołu JSON-RPC, który przesyła dane między stronami komunikacji za pomocą obiektów JSON. Pierwszą wiadomością w trakcie połączenia jest wymiana możliwości zarówno klienta (np. czy edytor wspiera przemianowanie zmiennej), jak i serwera (np. wskazanie definicji danego symbolu, czy automatyczne uzupełnianie pisanego tekstu).

\chapter{Część kliencka rozszerzenia}
W przypadku edytora Visual Studio Code, serwer LSP nie komunikuje się bezpośrednio z edytorem, potrzebny jest dodatkowy adapter w postaci osobnego małego rozszerzenia. W dodatku nie wszystkie możliwości edytora są wspierane przez protokół. Część kliencka rozszerzenia zatem ma 2 funkcjonalności:

\begin{enumerate}
    \item Uruchomienie serwera i komunikacja z nim.
    \item Udostępnienie funkcjonalności które nie są wspierane przez protokół LSP.
\end{enumerate}

\section{Komunikacja z serwerem}
Ponieważ zarówno klient jak i serwer LSP są napisane w środowisku Node.js, możliwe było wykorzystanie bibliotek udostępnionych przez twórców edytora, przez co nawiązanie połączenia i jego obsługa ogranicza się do wskazania pliku serwera. Dodatkowo moduł serwera zostaje przy utworzeniu dodany do listy obiektów usuwanych przy zamknięciu rozszerzenia (np. zamknięte zostały wszystkie pliki Lua lub został wyłączony edytor).

\section{Dodatkowe funkcjonalności}
\begin{enumerate}
    \item kolorowanie składni
    \item rejestracja nowego języka
    \item obsługa ustawień
\end{enumerate}
\chapter{Implementacja serwera LSP}
Dysponując parserem kodu Lua, wystarczy odpowiednio przechodzić generowane przez niego drzewa, w celu odpowiedzi na poszczególne zapytania klienta. Punktem wejścia dla projektu będącego częścią niniejszej pracy jest artykuł \cite{lsp_sample} opisujący utworzenie prostego serwera LSP.

\begin{figure}[H]
    \centering
\begin{tabular}{|c|c|}
\hline
Nazwa metody & Kierunek komunikacji Klient - Serwer\\
\hline
Initialize & $\hookleftarrow$ \\   
\hline
Initialized & $\rightarrow$ \\
\hline
Shutdown & $\hookleftarrow$ \\   
\hline
Exit & $\rightarrow$ \\
\hline
\end{tabular}
\end{figure}

\section{Zapytanie \texttt{Initialize}}
\begin{lstlisting}[language=JavaScript, basicstyle=\fontsize{9}{10}\ttfamily]
interface InitializeParams {
    processId: number | null
    rootPath?: string | null
    rootUri: DocumentUri | null
    initializationOptions?: any
    capabilities: ClientCapabilities
    trace?: 'off' | 'messages' | 'verbose'
}

interface ClientCapabilities {
    workspace?: WorkspaceClientCapabilities
    textDocument?: TextDocumentClientCapabilities
    experimental?: any
}

interface WorkspaceClientCapabilities {
    applyEdit?: boolean
    workspaceEdit?: { documentChanges?: boolean }
    didChangeConfiguration?: { dynamicRegistration?: boolean }
    didChangeWatchedFiles?: { dynamicRegistration?: boolean }
    symbol?: {
        dynamicRegistration?: boolean
        symbolKind?: { valueSet?: SymbolKind[] }
    }
    executeCommand?: { dynamicRegistration?: boolean }
}

interface TextDocumentClientCapabilities {
    synchronization?: {
        dynamicRegistration?: boolean
        willSave?: boolean
        willSaveWaitUntil?: boolean
        didSave?: boolean
    }
    completion?: {
        dynamicRegistration?: boolean
        completionItem?: {
            snippetSupport?: boolean
            commitCharactersSupport?: boolean
            documentationFormat?: MarkupKind[]
        }
        completionItemKind?: { valueSet?: CompletionItemKind[] }
        contextSupport?: boolean		
    }
    hover?: {
        dynamicRegistration?: boolean
        contentFormat?: MarkupKind[]
    }
    signatureHelp?: {
        dynamicRegistration?: boolean
        signatureInformation?: { documentationFormat?: MarkupKind[] }
    }
    references?: { dynamicRegistration?: boolean }
    documentHighlight?: { dynamicRegistration?: boolean }
    documentSymbol?: {
        dynamicRegistration?: boolean
        symbolKind?: { valueSet?: SymbolKind[] }
    }
    formatting?: { dynamicRegistration?: boolean }
    rangeFormatting?: { dynamicRegistration?: boolean }
    onTypeFormatting?: { dynamicRegistration?: boolean }
    definition?: { dynamicRegistration?: boolean }
    codeAction?: { dynamicRegistration?: boolean }
    codeLens?: { dynamicRegistration?: boolean }
    documentLink?: { dynamicRegistration?: boolean }
	rename?: { dynamicRegistration?: boolean }
}
\end{lstlisting}

\begin{lstlisting}[language=JavaScript, basicstyle=\fontsize{9}{10}\ttfamily]
interface InitializeResult {
    capabilities: ServerCapabilities
}

interface ServerCapabilities {
    textDocumentSync?: TextDocumentSyncOptions | number
    hoverProvider?: boolean
    completionProvider?: CompletionOptions
    signatureHelpProvider?: SignatureHelpOptions
    definitionProvider?: boolean
    referencesProvider?: boolean
    documentHighlightProvider?: boolean
    documentSymbolProvider?: boolean
    workspaceSymbolProvider?: boolean
    codeActionProvider?: boolean
    codeLensProvider?: CodeLensOptions
    documentFormattingProvider?: boolean
    documentRangeFormattingProvider?: boolean
    documentOnTypeFormattingProvider?: DocumentOnTypeFormattingOptions
    renameProvider?: boolean
    documentLinkProvider?: DocumentLinkOptions
    executeCommandProvider?: ExecuteCommandOptions
    experimental?: any
}

namespace TextDocumentSyncKind {
     const None = 0
     const Full = 1
     const Incremental = 2
}

interface CompletionOptions {
    resolveProvider?: boolean
    triggerCharacters?: string[]
}

interface SignatureHelpOptions {
    triggerCharacters?: string[]
}

interface CodeLensOptions {
    resolveProvider?: boolean
}

interface DocumentOnTypeFormattingOptions {
    firstTriggerCharacter: string
    moreTriggerCharacter?: string[]
}

interface DocumentLinkOptions {
    resolveProvider?: boolean
}

interface ExecuteCommandOptions {
    commands: string[]
}

interface SaveOptions {
    includeText?: boolean
}

interface TextDocumentSyncOptions {
    openClose?: boolean
    change?: number
    willSave?: boolean
    willSaveWaitUntil?: boolean
    save?: SaveOptions
}
\end{lstlisting}
W odpowiedzi na to zapytanie serwer zwraca informacje na temat jego możliwości. W przypadku tej pracy są to:

\begin{itemize}
    \item Znalezienie definicji danej zmiennej lub funkcji.
    \item Automatyczne sugestie pisanego kodu.
    \item Wyświetlenie informacji na temat danego symbolu.
\end{itemize}

\section{Komunikaty generujące drzewa}
\subsection{Komunikat \texttt{DidChangeTextDocument}}
Komunikat \texttt{DidChangeTextDocument} informuje serwer, że użytkownik dokonał zmian w danym pliku. Uruchomiony zostaje wtedy parser, który czyta treść rzeczonego pliku (możliwością jest pracowanie na samej treści zmiany, jednakże wiązałoby się to z koniecznością napisania własnego parsera wspierającego inkrementalne zmiany w parsowanym tekście). Jeżeli parser napotkał jakiś błąd, jest on zwracany z powrotem do klienta i wyświetlany użytkownikowi pod postacią czerwonego podkreślenia problematycznego fragmentu kodu. Przy udanym parsowaniu otrzymane drzewo jest zapisywane w pamięci, aby można było je odwiedzić przy odpowiadaniu na inne zapytania.

\subsection{Komunikat \texttt{DicChangeConfiguration}}
Komunikat \texttt{DidChangeConfiguration} informuje nas, że użytkownik zmienił ustawienia edytora, co mogło wpłynąć na pliki w niekontrolowany przez nas sposób. W takim przypadku następuje ponowne parsowanie wszystkich otwartych dokumentów analogicznie do komunikatu \texttt{DidChangeTextDocument}.

\section{Zapytania przechodzące po drzewie}
\subsection{Funkcja \texttt{TraverseTreeDown}}
Każde z zapytań które mają w efekcie przejść się po drzewie rozbioru dostarcza nam informacje na temat pozycji w pliku na której się znajduje kursor. W takim razie wydzielona została funkcjonalność tłumaczenia pozycji na węzeł drzewa. Ponieważ każdy z węzłów drzewa rozbioru zawiera informację na temat zakresu tegoż węzła, wystarczy przejść się wgłąb drzewa tak długo jak szukana pozycja znajduje się wewnątrz zakresu przeszukiwanego wierzchołka. Funkcja zwraca odnaleziony wierzchołek.

\subsection{Zapytanie \texttt{Hover}}
Zapytanie \texttt{Hover} pyta się o możliwe informacje do wyświetlenia gdy użytkownik najedzie kursorem myszy na daną pozycję w pliku. Po znalezieniu najlepiej pasującego wierzchołka za pomocą funkcji \texttt{TraverseTreeDown}, zwracany jest komunikat którego dokładna treść jest określana na podstawie typu wierzchołka (np. dla funkcji będzie to jej nazwa i lista argumentów, a dla zmiennej jej typ).

\subsection{Zapytanie \texttt{GotoDefinition}}
Zapytanie \texttt{GotoDefinition} odpowiada akcji polegającej na szukaniu miejsca zdefiniowania danego symbolu w kodzie. Po znalezieniu szukanego symbolu następuje ponowne przejście po drzewie w celu odnalezienia najpóźniejszej definicji tegoż symbolu (w języku Lua symbole mogą być definiowane na nowo w trakcie działania programu, a także przesłaniane za pomocą słowa kluczowego \texttt{local}). Zwracana jest pozycja odnalezionej definicji.

\subsection{Zapytanie \texttt{Completion}}
Zapytanie \texttt{Completion} jest wysyłane przez klienta w celu odpytania serwera na temat możliwego dokończenia aktualnie pisanego tekstu. Również i w tym przypadku dostarczana jest pozycja kursora, jednakże serwer nie szuka aktualnie edytowanego wierzchołka, tylko listę symboli które zostały zdefiniowane i są dostępne w danym kontekście. Lista odnalezionych symboli jest później rozszerzana o funkcje i zmienne zdefiniowane w bibliotece standardowej Lua (informacje na ich temat znajdują się w osobnym pliku JSON, który został utworzony na podstawie dokumentacji języka \cite{lua_lib}).
\chapter{Podsumowanie}
W ramach niniejszej pracy powstało rozszerzenie programu Visual Studio Code, które wspomaga programistę przy pisaniu kodu. Główna część programu, mianowicie serwer LSP, może zostać użyta przy implementacji analogicznego rozszerzenia dla innych edytorów, bez potrzeby wprowadzania zmian w kodzie. 
Powstałe rozszerzenie implementuje znaczną część funkcjonalności protokołu LSP, dodanie do niego pozostałych funkcji nie będzie zadaniem trudnym, jedynie czasochłonnym. Zadanie było rozwijające zarówno pod względem pracy z parserem kodu i interpretowaniem jego wyników, ale również pozwoliło prześledzić cały proces rozszerzania funkcjonalności istniejącego programu za pomocą udostępnionego interfejsu.

\begin{thebibliography}{1}
\bibitem{tag} authors \href{link}{title}, year.
\bibitem{docs} Microsoft \href{https://microsoft.github.io/language-server-protocol/specification#initialize}{Language Server Protocol documentation}, 2018.
\bibitem{parse} Oskar Schöldström \href{https://oxyc.github.io/luaparse/}{Luaparse}, 2013
\end{thebibliography}

\renewcommand\appendixtocname{Dodatki}
\bookmarksetupnext{level=part}
\begin{appendices}
\addtocontents{toc}{\protect\setcounter{tocdepth}{1}}
\makeatletter
\addtocontents{toc}{
  \begingroup
  \let\protect\l@chapter\protect\l@section
  \let\protect\l@section\protect\l@subsection
}
\makeatother
  \chapter{Instrukcja uruchomienia rozszerzenia}
Niniejszy dodatek opisuje instrukcję uruchomienia rozszerzenia w programie Visual Studio Code. Wymagana jest dodatkowo instalacja środowiska Node.js, które musi być dostępne z wiersza poleceń. Kroki do wykonania:

\begin{enumerate}
    \item Otworzyć katalog \textbf{lua} w edytorze VS Code.
    \item Nacisnąć kombinację klawiszy \texttt{Ctrl+`}, otwierając tym samym wbudowane okno terminala.
    \item Wykonać polecenie \texttt{npm install}, które zainstaluje brakujące pakiety od których zależy rozszerzenie.
    \item W sekcji \textbf{Debug} wybrać konfigurację \textbf{Launch Client} i nacisnąć przycisk zielonej strzałki (ewentualnie nacisnąć klawisz \texttt{F5}).
    \item Zostanie otwarta druga instancja edytora, w której rozszerzenie jest aktywne.
\end{enumerate}
\addtocontents{toc}{\endgroup}
\end{appendices}

\end{document}